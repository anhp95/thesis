\chapter{Ukraine's case study}
\section{Introduction}
Nitrogen dioxide (NO\textsubscript{2}) is a key air pollutant that can have harmful effects on human health. An increase in nitrogen oxide (NOx = NO + NO2) concentrations contributes to global warming through a chemical reaction that leads to the formation of ozone (O3), a short- lived climate pollutant with a potent warming effect \citep{ipcc2013}. The lifetime of NO2 is strongly influenced by photochemical reactions and meteorological parameters \citep{barre2021estimating} and varies seasonally \citep{dragomir2015modeling,kendrick2015diurnal}. During winter, photochemical reaction activity is reduced, resulting in a longer lifetime of the NO2. Additionally, seasonal variations in NO2 concentration are controlled by dispersion processes which are significantly affected by changes in boundary layer height (BLH), wind speed and direction patterns due to temperature inversions in summer and winter \citep{barre2021estimating,kendrick2015diurnal}. NO2 concentration levels have been widely used to evaluate decreases in emissions associated with intervention events such as the COVID-19 pandemic lockdown and impacts on the air quality due to the short lifetime of NO2 in the atmosphere \citep{barre2021estimating,cooper2022global}.
In Europe, anthropogenic NOx emissions are mainly attributed to combustion processes in transportation, as well as energy production and distribution. \par
In Ukraine, coal-fired power plants (CPPs) dominantly account for 80\% of total SO2 and 25\% of total NOx emissions, and some have been identified as the highest-emitting CPPs in the region and in the world \citep{lauri2021}. Since the pandemic started in March 2020, and now with the ongoing armed conflict with Russia, Ukraine has faced a series of threats to the economy, human security and the environment, as well as geopolitical tensions \citep{pereira2022russian}. During the pandemic response starting in 2020, many national and local lockdown restrictions were issued to prevent the spread of the virus, causing a sharp decrease in gross domestic product growth rate, as well as industrial and energy production \citep{danylyshyn2020peculiarities}. In 2021, Ukraine’s economy started to recover from the pandemic but the recovery was eventually upended by an armed conflict with Russia that started on February 24, 2022. The conflict has been causing a multi-pronged crisis not only in Ukraine but also in Europe, with increased prices and exacerbated inflation among the many impacts. Many facilities and extensive areas of housing and other infrastructure, including some CPPs, have been reported destroyed or damaged in Ukraine. These impacts have consequently triggered an unprecedented refugee crisis in Ukraine, clogging border crossings between Ukraine and bordering European countries \citep{julia2022}. The many socio-economic changes that have occurred during the pandemic and the conflict could be expected to contribute to major variability in air quality in Ukraine, including NO2 pollution levels, during the 2020–2022 period. \par
A report by the United Nations Development Programme (UNDP) \citep{dumitru2020}, estimated the impacts of the pandemic lockdown on NO2 levels in Ukraine by using Sentinel 5P (S5P) NO2 column concentrations and Copernicus Atmosphere Monitoring Service (CAMS) surface NO2 data (Marécal et al., Citation2015). However, meteorological variables were not acknowledged, although ignoring weather factors could strongly affect final estimates of changes in pollution concentration levels induced by the lockdown (Schiermeier, Citation2020). A more recent study \citep{zalakeviciute2022war} utilized direct satellite observation from 2019 and early 2020 as business-as-usual data to evaluate the impact of the Russia-Ukraine conflict in 2022 on air quality, but again, without acknowledging weather effects. These two studies utilized estimates of year-to-year differences. However, such estimates can easily be affected and dominated by changes in meteorological parameters rather than emission sources \citep{grange2021covid,shi2021abrupt}. Therefore, a more sophisticated method is needed to measure the impacts of intervention events through better quantification of actual air quality. \par
In order to normalize the meteorological effects to accurately and reliably quantify the impact of intervention events, the use of machine learning is increasingly being adopted, but mostly applied for ground-based measurements following the original idea proposed by \citep{grange2018random} and \citep{grange2019using}. The objective of this approach is to construct a business-as-usual (BAU) model for predicting air pollution levels independently of the impacts of any intervention events. This is achieved by integrating meteorological, spatial, and temporal features into the model during the BAU period to accurately represent air pollution levels. An intervention event, in this context, refers to an occurrence that has caused changes in air quality. Recently, \citep{barre2021estimating} have introduced their weather normalization approach to improve estimates of lockdown impacts not only on NO2 levels from ground-based observations and CAMS simulations, but also in satellite measurements from S5P. The original method in \citep{grange2018random,grange2019using} has been altered in order to work with satellite retrieval column NO2 concentration levels from S5P by adopting a new feature, the forecast surface NO2 level from CAMS data. Alternatively, gradient boosting machines (GBMs) \citep{friedman2001greedy} have been also utilized instead of random forests \citep{grange2018random} to develop weather-normalization models under the BAU conditions. \citep{barre2021estimating} reported an overall reduction (ranging from 23\% to 32\%) in major European cities using the three datasets. Their study showed an average difference of 14\% between satellite-based and ground-based estimates, and 11\% between simulations from the CAMS regional ensemble of air quality models and ground-based estimates. These findings suggest that estimates of the impacts of the lockdown on NO2 levels can vary depending on the source of the data.\par
This chapter aims to investigate the actual satellite-derived column NO2 pollution levels induced by pandemic lockdown restrictions and the armed conflict with Russia, which have been two major changes in human activities in Ukraine since 2019. In order to do so, we developed a weather-normalization model under BAU scenarios for S5P column NO2 levels to decouple the meteorological effects from the intervention effects. The BAU simulation NO2 levels are then used to quantify changes in S5P column NO2 concentrations during the lockdown and the armed conflict. We describe the data used in the study in Section 2 and the methodology in Section 3. The results and discussion on NO2 level changes are summarized in Section 4 for the lockdown, and Section 5 for the armed conflict. Finally, we conclude the results of the study in Section 6.\par
\section{Data}
\section{Business-as-usual (BAU) modelling}
\section{NO2 changes induced by COVID-19 lockdown}
\section{NO2 changes induced by the armed conflict}
\section{Conclusion}