\chapter{Ukraine's case study}
\section{Introduction}
Nitrogen dioxide (NO\textsubscript{2}) is a key air pollutant that can have harmful effects on human health. An increase in nitrogen oxide (NOx = NO + NO2) concentrations contributes to global warming through a chemical reaction that leads to the formation of ozone (O3), a short- lived climate pollutant with a potent warming effect \citep{ipcc2013}. The lifetime of NO2 is strongly influenced by photochemical reactions and meteorological parameters \citep{barre2021estimating} and varies seasonally \citep{dragomir2015modeling,kendrick2015diurnal}. During winter, photochemical reaction activity is reduced, resulting in a longer lifetime of the NO2. Additionally, seasonal variations in NO2 concentration are controlled by dispersion processes which are significantly affected by changes in boundary layer height (BLH), wind speed and direction patterns due to temperature inversions in summer and winter \citep{barre2021estimating,kendrick2015diurnal}. NO2 concentration levels have been widely used to evaluate decreases in emissions associated with intervention events such as the COVID-19 pandemic lockdown and impacts on the air quality due to the short lifetime of NO2 in the atmosphere \citep{barre2021estimating,cooper2022global}.
In Europe, anthropogenic NOx emissions are mainly attributed to combustion processes in transportation, as well as energy production and distribution. \par
In Ukraine, coal-fired power plants (CPPs) dominantly account for 80\% of total SO2 and 25\% of total NOx emissions, and some have been identified as the highest-emitting CPPs in the region and in the world \citep{lauri2021}. Since the pandemic started in March 2020, and now with the ongoing armed conflict with Russia, Ukraine has faced a series of threats to the economy, human security and the environment, as well as geopolitical tensions \citep{pereira2022russian}. During the pandemic response starting in 2020, many national and local lockdown restrictions were issued to prevent the spread of the virus, causing a sharp decrease in gross domestic product growth rate, as well as industrial and energy production \citep{danylyshyn2020peculiarities}. In 2021, Ukraine’s economy started to recover from the pandemic but the recovery was eventually upended by an armed conflict with Russia that started on February 24, 2022. The conflict has been causing a multi-pronged crisis not only in Ukraine but also in Europe, with increased prices and exacerbated inflation among the many impacts. Many facilities and extensive areas of housing and other infrastructure, including some CPPs, have been reported destroyed or damaged in Ukraine. These impacts have consequently triggered an unprecedented refugee crisis in Ukraine, clogging border crossings between Ukraine and bordering European countries \citep{julia2022}. The many socio-economic changes that have occurred during the pandemic and the conflict could be expected to contribute to major variability in air quality in Ukraine, including NO2 pollution levels, during the 2020–2022 period.
\section{Data}
\section{Business-as-usual (BAU) modelling}
\section{NO2 changes induced by COVID-19 lockdown}
\section{NO2 changes induced by the armed conflict}
\section{Conclusion}