\chapter{Conclusions and future prospects}\chaplab{conclusion} \label{chap6}
\section{Key finding and contributions}
In this study, my initial focus was on researching air pollution at the local level. I investigated the impact of extreme events on regional air pollution changes, aiming to provide evidence and recommendations for future local policies. Subsequently, my attention shifted to GHGs monitoring. Specifically, I worked on estimating global terrestrial carbon fluxes using updated PFTs data and Transformer-based models which have been limitedly adopted in previous studies. Finally, I developed a digital earth platform to monitor fossil fuel GHGs as well as the capacity of terrestrial forest carbon fluxes, enabling the development of efficient carbon neutrality roadmaps at the local level. In this section of the study, I address the five research questions raised in Chapter \ref{chap1}, with references to the chapters and their relevant contributions. \par

\textit{1. What was the influence of the COVID-19 lockdown and the armed conflict on air quality in Ukraine, and how can this information offer evidence and recommendations for future policies?} \par
I analyzed NO\textsubscript{2} levels in Ukraine during two significant periods and found that meteorological variations were the primary contributors to the reduction in NO\textsubscript{2} in populous cities during the lockdown period in 2020. After normalizing for meteorological effects, we observed a moderation in the increase of NO\textsubscript{2} levels during the lockdown compared to pre-lockdown levels. Examining the same months during the conflict in 2022, we identified even more substantial reductions in NO\textsubscript{2} levels in these cities. Additionally, beyond our investigation of major urban areas, we noticed decreases in NO\textsubscript{2} levels in areas surrounding coal power plants that were damaged or destroyed during the conflict. Regarding major urban areas in Ukraine, we conclude that changes in daily anthropogenic activities due to conflict-related events had a more significant impact on NO\textsubscript{2} levels than the COVID-19 lockdown. \par

In Europe, the primary source of NO\textsubscript{2} emissions is road transport \citep{aq2020eu}. However, in Ukraine, most NO\textsubscript{2} emissions originate from Coal Power Plants (CPPs) \citep{lauri2021}. The lockdown led to a significant reduction in NO\textsubscript{2} emissions in European cities due to decreased mobility \citep{barre2021estimating}. Contrary to this, the reduction was not apparent in Ukraine after accounting for weather effects, likely due to limited decreases in the operation of Coal Power Plants (CPPs), as outlined in our study in Chapter \ref{chap3}. However, during the armed conflict with Russia, many CPPs in Ukraine were damaged, resulting in a noticeable reduction in NO\textsubscript{2} emissions in populous cities. This serves as evidence that future policies aimed at reducing NO\textsubscript{2} from road transportation may not be as effective in Ukraine's populous cities. Detailed results for this analysis are presented in Chapter \ref{chap3}, section \ref{chap3_s1}. \par

\textit{2. In what ways did the COVID-19 lockdown influence air quality in Japan, and how can this information serve as evidence and provide suggestions for future policies?} \par
I investigated the impact of NO\textsubscript{2} reduction on O\textsubscript{3} and CH\textsubscript{4} in 14 metropolitan areas (MAs) of Japan in 2020 by employing business-as-usual air quality time series generated by machine learning models. Additionally, I use satellite observations and biogeochemical model simulations to analyse air quality changes. I found that during the lockdown period from April 7 to May 25 in 2020, I observed a NO\textsubscript{2} reduction that equated to a decrease equivalent to 3.4 years and 5 years of the corresponding trends in roadside and ambient air quality recorded from 2010 to 2019. After meteorological normalization, NO\textsubscript{2} decreased by 14.5\% at ambient air stations and 19.1\% at roadside stations. Surprisingly, the NO\textsubscript{2} reduction did not immediately lead to increased O\textsubscript{3}. Instead, O\textsubscript{3} levels rose after the lockdown, specifically in August due to favourable sunny conditions. This finding is important for Japan and has not been reported in previous studies. We found that changes in NO\textsubscript{2} and CO marginally contributed to variations in CH\textsubscript{4} levels across the study areas. \par

We present evidence indicating an increase in O\textsubscript{3} after the lockdown in most of the Monitoring Areas (MAs) from Okayama northwards. This occurrence could potentially be attributed to these MAs being VOC-limited areas, suggesting that future reductions in NO\textsubscript{2} may lead to increased O\textsubscript{3} levels under favorable sunny conditions. However, in MAs from Hiroshima southwards, instances of NO\textsubscript{x} limitation were observed, implying that future reductions in anthropogenic non-methane volatile organic compounds (NMVOCs) may have minimal effectiveness in lowering O\textsubscript{3} levels \citep{akimoto2022rethinking}. Therefore, to effectively mitigate the adverse effects on O\textsubscript{3}, it is recommended to simultaneously reduce air pollutants, as well as anthropogenic and biogenic volatile organic compounds, in future policies. Detailed results for this analysis are presented in Chapter \ref{chap3}, section \ref{chap3_s2}. \par

\textit{3. What methodologies can be employed to improve PFTs mapping performance in data-sparse regions?} \par
We proposed a combined machine learning approach with a deep convolutional neural network (CNN) which improves the accuracy of PFTs mapping and tree age estimation in Ena city, Japan. First, we employed the Random Forest (RF) classifier using Google Earth Engine (GEE) for forest mapping. Then, we designed a deep CNN architecture that works for PFTs and forest age mapping from coarse and polygonal ground-truth data. The proposed network has U-shape and comprises 3D Atrous Convolutions. The model was optimized by a weighted cross-entropy loss function. We trained the model with times-series Sentinel 1, 2, and Digital Elevation Model (DEM) data with sparse annotations. Our proposed models achieved 94.5\% overall accuracy (OA) for forest mapping, 77.80\% (OA) for PFTs, and 81.74\% (OA) for forest age classification, respectively which outperformed the 2D and 3D UNET performance. Detailed results for this analysis are presented in Chapter \ref{chap4}, section \ref{chap4_s1}. \par
\textit{4. Can the utilization of updated PFT maps and Transformer-based models enhance the accuracy of global terrestrial carbon flux estimates?} \par
Yes, by utilizing the new PFTs dataset in combination with MVTS Transformer-based model we provided a monthly global gross primary production and ecosystem respiration dataset from 1990 to 2019 at 0.25° × 0.25° spatial resolution which outperforms FLUXCOM, NIES, and MetaFlux datasets when comparing the correlation at site-level and seasonal pattern with FLUXNET 2015, especially in tropical regions. Additionally, our dataset reveals the highest positive trend in GPP from 2001 to 2019, aligning with studies like \citep{{piao2020characteristics, guo2023estimating, yang2022divergent}}. Notably, it captures long-term trends that FLUXCOM and MetaFlux fail to replicate, contradicting the observed significant greening reported by \citep{piao2020characteristics}. Lastly, we compare our dataset's interannual variations with other datasets, finding lower variations in extreme-low-GPP regions than NIES data when considering the same utilized remote sensing resources. Detailed results for this analysis are presented in Chapter \ref{chap4}, section \ref{chap4_s2}. \par
\textit{5. How can we efficiently monitor fossil fuels GHGs emissions as well as the capacity of terrestrial forest carbon fluxes, enabling the development of efficient carbon neutrality roadmaps at the local level?} \par
We have developed a digital earth platform for monitoring greenhouse gas emissions from fossil fuels, offering a roadmap for achieving carbon neutrality at the municipality level in Japan. Our platform integrates energy-related data, including information on energy consumption and electricity statistics from major domestic power companies. This encompasses data on electricity usage, forecasts, and supply, along with an assessment of the capacity of forest sinks. This integrated information provides a comprehensive overview of the current status towards achieving zero-carbon at the municipality level in Japan. The platform is accessible at the following URL: \url{http://de14.digitalasia.chubu.ac.jp/}. Detailed results for this analysis are presented in Chapter \ref{chap5}.\par

\section{Future prospects}
CO\textsubscript{2} stands out as a crucial greenhouse gas, but monitoring fossil fuel CO\textsubscript{2} emissions in near real time remains challenging, leading to high uncertainties in estimated results \citep{marland2008uncertainties}. Traditional bottom-up inventories are time-consuming \citep{marland2008uncertainties}. Recently, a top-down method has emerged, leveraging advancements in satellite observations and data assimilation frameworks. However, current satellites like GOSAT and OCO-2 were designed to focus on the spatiotemporal distribution of natural carbon fluxes at regional scales, rather than quantifying anthropogenic emissions \citep{nassar2017quantifying, yang2023using}. Consequently, the spatial and temporal limitations of these CO\textsubscript{2} observations hinder their ability to estimate CO\textsubscript{2} emissions at the urban or city levels. \par

Conversely, existing long-term satellite-derived NO\textsubscript{2} observations, such as OMI or TROPOMI, exhibit more advanced capabilities with higher resolutions in spatiotemporal aspects. They hold the potential to serve as instruments in constraining fossil fuel CO\textsubscript{2} emissions at city levels. Thus, an indirect top-down method harnesses proxies like NO\textsubscript{2} observations, given their co-emission with fossil fuel CO\textsubscript{2} combustion. This indirect method proves beneficial in constraining fossil CO\textsubscript{2} emissions, monitoring their temporal fluctuations, while distinguishing them from biogenic sources of CO\textsubscript{2} emission itself \citep{ciais2014current, goldberg2019exploiting}. Satellite-based NO\textsubscript{2} observations, combined with NO\textsubscript{x}:CO\textsubscript{2} inventory ratios, have been instrumental in estimating CO\textsubscript{2} emissions indirectly. These approaches have been applied at national scales in countries such as the US, Europe, China, and India \citep{konovalov2016estimation, zheng2020satellite, miyazaki2023predictability} and at city levels, such as in Wuhan \citep{zhang2023quantifying} Buenos Aires, Melbourne, and Mexico City \citep{yang2023using}. However, such analyses have not yet been conducted either at the national or municipal levels in Japan. Conducting studies employing these methodologies both at national and cities levels in Japan could provide supplemental independent datasets. These datasets would serve to refine and evaluate "bottom-up" inventories and to assess the efficacy of current climate change mitigation strategies related to reducing fossil fuel CO\textsubscript{2} emissions, contributing insights from local to global scales. Therefore, such investigations are necessary and could offer valuable information to refine our understanding of CO\textsubscript{2} emissions and strategies for mitigating climate change. \par

Additionally, leveraging the high-frequency temporal data from satellite-derived NO\textsubscript{2} observations for predicting fossil fuel CO\textsubscript{2}, along with the global upscaled carbon fluxes detailed in Chapter \ref{chap4}, section \ref{chap4_s2}, is anticipated to provide the most current assessment of achieving zero carbon emissions on both regional and global scales when integrated into a Digital Earth platform as described in Chapter \ref{chap5}. Recent publications in this field within the current year, such as \citep{zhang2023quantifying, yang2023using, miyazaki2023predictability}, reflect active research in these areas. This also presents a room for future researchs in my work.\par