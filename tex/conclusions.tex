\chapter{Conclusion}\chaplab{conclusion} \label{chap8}
\section{Key finding and contributions}
In this study, my initial emphasis was on researching air pollution at the local level. I examined the effects of extreme events on regional air pollution changes and drew lessons that could inform future policies. Subsequently, my focus shifted to greenhouse gas monitoring research. Specifically, I concentrated on estimating global terrestrial carbon fluxes and the development of a platform to monitor emissions of greenhouse gases from fossil fuels, along with examining carbon sequestration in forests and considering other pertinent factors at the local level. In this section I answer these 5 research questions that are raised in chapter \ref{chap1}, with references to the chapters and related contributions. \par

\textit{1. How did the COVID-19 lockdown and the armed conflict impact air quality in Ukraine, and what lessons can be derived for future policies?} \par
I analyzed NO\textsubscript{2} levels in Ukraine during two significant periods and determined that meteorological factors were the primary contributors to the reduction in NO\textsubscript{2} in populous cities during the lockdown period in 2020. After normalizing for meteorological effects, we observed a moderation in the increase of NO\textsubscript{2} levels during the lockdown compared to pre-lockdown levels. Examining the same months during the conflict in 2022, we identified even more substantial reductions in NO\textsubscript{2} levels in these cities. Additionally, beyond our investigation of major urban areas, we noticed decreases in NO\textsubscript{2} levels in areas surrounding coal power plants that were damaged or destroyed during the conflict. Regarding major urban areas in Ukraine, we conclude that changes in daily anthropogenic activities due to conflict-related events had a more significant impact on NO\textsubscript{2} levels than the COVID-19 lockdown. We recommend adopting a more stringent approach in future policies to reduce NO\textsubscript{2} levels in Ukraine's urban areas. Detailed results for this analysis are presented in Chapter \ref{chap3}. \par

\textit{2. In what ways did the COVID-19 lockdown influence air quality in Japan, and what lessons can be learned for future policy considerations?} \par
I investigated the impact of NO\textsubscript{2} reduction on O\textsubscript{3} and CH\textsubscript{4} in 14 metropolitan areas of Japan in 2020 by employing business-as-usual air quality time series generated by machine learning models. Additionally, I use satellite observations and biogeochemical model simulations to analyse air quality changes. I found that during the lockdown period from April 7 to May 25 in 2020, I observed a NO\textsubscript{2} reduction that equated to a decrease equivalent to 3.4 years and 5 years of the corresponding trends in roadside and ambient air quality recorded from 2010 to 2019. After meteorological normalization, NO\textsubscript{2} decreased by 14.5\% at ambient air stations and 19.1\% at roadside stations. Surprisingly, the NO\textsubscript{2} reduction did not immediately lead to increased O\textsubscript{3}. Instead, O\textsubscript{3} levels rose after the lockdown, specifically in August due to favourable sunny conditions. This finding is important for Japan and has not been reported in previous studies. We found that changes in NO\textsubscript{2} and CO marginally contributed to variations in CH\textsubscript{4} levels across the study areas. To effectively mitigate the adverse effects on O\textsubscript{3} and CH\textsubscript{4}, it is recommended to simultaneously reduce air pollutants as well as anthropogenic and biogenic volatile organic compounds in future policies. Detailed results for this analysis are presented in Chapter \ref{chap4}. \par

\textit{3. What methodology can be employed to map Plant Functional Types (PFTs) in data-sparse regions?} \par
We proposed a combined machine learning approach with a deep convolutional neural network (CNN) which improves the accuracy of PFTs mapping and tree age estimation in Ena city, Japan. First, we employed the Random Forest (RF) classifier using Google Earth Engine (GEE) for forest mapping. Then, we designed a deep CNN architecture that works for PFTs and forest age mapping from coarse and polygonal ground-truth data. The proposed network has U-shape and comprises 3D Atrous Convolutions. The model was optimized by a weighted cross-entropy loss function. We trained the model with times-series Sentinel 1, 2, and Digital Elevation Model (DEM) data with sparse annotations. Our proposed models achieved 94.5\% overall accuracy (OA) for forest mapping, 77.80\% (OA) for PFTs, and 81.74\% (OA) for forest age classification, respectively which outperformed the 2D and 3D UNET performance. Detailed results for this analysis are presented in Chapter \ref{chap5}. \par
\textit{4. Can the utilization of updated PFT maps and models based on Transformer architecture enhance the accuracy of global carbon flux estimates?}
Yes, by utilizing the new PFTs dataset in combination with MVTS Transformer-based model we provided a monthly global gross primary production and ecosystem respiration dataset from 1990 to 2019 at 0.25° × 0.25° spatial resolution which outperforms FLUXCOM, NIES, and MetaFlux datasets when comparing the correlation at site-level and seasonal pattern with FLUXNET 2015, especially in tropical regions. Additionally, our dataset reveals the highest positive trend in GPP from 2001 to 2019, aligning with studies like \citep{{piao2020characteristics, guo2023estimating, yang2022divergent}}. Notably, it captures long-term trends that FLUXCOM and MetaFlux fail to replicate, contradicting the observed significant greening reported by \citep{piao2020characteristics}. Lastly, we compare our dataset's interannual variations with other datasets, finding lower variations in extreme-low-GPP regions than NIES data when considering the same utilized remote sensing resources. Detailed results for this analysis are presented in Chapter \ref{chap6}. \par
\textit{5. How can we efficiently monitor emissions of greenhouse gases derived from fossil fuels and the carbon sequestration from forests, in addition to addressing other relevant factors at the local level?}
We have developed a digital earth platform for monitoring greenhouse gas emissions from fossil fuels, offering a roadmap for achieving carbon neutrality at the municipality level in Japan. Our platform integrates energy-related data, including information on energy consumption and electricity statistics from major domestic power companies. This encompasses data on electricity usage, forecasts, and supply, along with an assessment of the capacity of forest sinks. This integrated information provides a comprehensive overview of the current status towards achieving zero-carbon at the municipality level in Japan. The platform is accessible at the following URL: \url{http://de14.digitalasia.chubu.ac.jp/}. Detailed results for this analysis are presented in Chapter \ref{chap7}.\par

\section{Future prospects}
CO\textsubscript{2} stands out as a crucial greenhouse gas, but monitoring fossil fuel CO\textsubscript{2} emissions in near real time remains challenging, leading to high uncertainties in estimated results \citep{marland2008uncertainties}. Traditional bottom-up inventories are time-consuming \citep{marland2008uncertainties}. Recently, a top-down method has emerged, leveraging advancements in satellite observations and data assimilation frameworks. However, current satellites like GOSAT and OCO-2 were designed to focus on the spatiotemporal distribution of natural carbon fluxes at regional scales, rather than quantifying anthropogenic emissions \citep{nassar2017quantifying, yang2023using}. Consequently, the spatial and temporal limitations of these CO\textsubscript{2} observations hinder their ability to estimate CO\textsubscript{2} emissions at the urban or city levels. \par

Conversely, existing long-term satellite-derived NO\textsubscript{2} observations, such as OMI or TROPOMI, exhibit more advanced capabilities with higher resolutions in spatiotemporal aspects. They hold the potential to serve as instruments in constraining fossil fuel CO\textsubscript{2} emissions at city levels. Thus, an indirect top-down method harnesses proxies like NO\textsubscript{2} observations, given their co-emission with fossil fuel CO\textsubscript{2} combustion. This indirect method proves beneficial in constraining fossil CO\textsubscript{2} emissions, monitoring their temporal fluctuations, while distinguishing them from biogenic sources of CO\textsubscript{2} emission itself \citep{ciais2014current, goldberg2019exploiting}. Satellite-based NO\textsubscript{2} observations, combined with NO\textsubscript{x}:CO\textsubscript{2} inventory ratios, have been instrumental in estimating CO\textsubscript{2} emissions indirectly. These approaches have been applied at national scales in countries such as the US, Europe, China, and India \citep{konovalov2016estimation, zheng2020satellite, miyazaki2023predictability} and at city levels, such as in Wuhan \citep{zhang2023quantifying} Buenos Aires, Melbourne, and Mexico City \citep{yang2023using}. However, such analyses have not yet been conducted either at the national or municipal levels in Japan. Conducting studies employing these methodologies both at national and cities levels in Japan could provide supplemental independent datasets. These datasets would serve to refine and evaluate "bottom-up" inventories and to assess the efficacy of current climate change mitigation strategies related to reducing fossil fuel CO\textsubscript{2} emissions, contributing insights from local to global scales. Therefore, such investigations are necessary and could offer valuable information to refine our understanding of CO\textsubscript{2} emissions and strategies for mitigating climate change. \par

Additionally, leveraging the high-frequency temporal data from satellite-derived NO\textsubscript{2} observations for predicting fossil fuel CO\textsubscript{2}, along with the global upscaled carbon fluxes detailed in Chapter \ref{chap6}, is anticipated to provide the most current assessment of achieving zero carbon emissions on both regional and global scales when integrated into a Digital Earth platform as described in Chapter \ref{chap7}. Recent publications in this field within the current year, such as \citep{zhang2023quantifying, yang2023using, miyazaki2023predictability}, reflect active research in these areas. This also presents a room for future researchs in my work.\par