\chapter{Conclusions and future prospects}\chaplab{conclusion} \label{chap6}
\section{Key finding and contributions}
In this study, my initial focus was on researching air pollution at the local level. I investigated the impact of intervention events on regional air pollution changes, aiming to provide evidence and recommendations for future local policies. Subsequently, my attention shifted to GHGs monitoring. Specifically, I worked on estimating global terrestrial carbon fluxes using updated PFTs data and Transformer-based models which have been limitedly adopted in previous studies. Finally, I developed a digital earth platform to monitor fossil fuel GHGs as well as the capacity of terrestrial forest carbon fluxes, enabling the development of efficient carbon neutrality roadmaps at the local level. In this section of the study, I address the five research questions raised in Chapter \ref{chap1}, with references to the chapters and their relevant contributions. \par

\textit{1. What was the influence of the COVID-19 lockdown and the armed conflict on air quality in Ukraine, and how can this information offer evidence and recommendations for future policies?} \par
I analyzed tropospheric NO\textsubscript{2} levels in Ukraine during two significant periods and found that meteorological variations were the primary contributors to the reduction in NO\textsubscript{2} in populous cities during the lockdown period in 2020. After normalizing for meteorological effects, I observed a moderation in the increase of NO\textsubscript{2} levels during the lockdown compared to pre-lockdown levels. Examining the same months during the conflict in 2022, we identified even more substantial reductions in NO\textsubscript{2} levels in these cities. Additionally, beyond our investigation of major urban areas, we noticed decreases in NO\textsubscript{2} levels in areas surrounding coal power plants that were damaged or destroyed during the conflict. Regarding major urban areas in Ukraine, we conclude that changes in daily anthropogenic activities due to conflict-related events had a more significant impact on NO\textsubscript{2} levels than the COVID-19 lockdown. \par

In Europe, road transport is identified as the primary contributor to NO\textsubscript{2} emissions \citep{aq2020eu}. Conversely, in Ukraine, the major source of NO\textsubscript{2} emissions is traced back to Coal Power Plants (CPPs) \citep{lauri2021}. The imposition of COVID-19 lockdown measures in 2020 resulted in a substantial decrease in NO\textsubscript{2} emissions in European cities due to reduced mobility \citep{barre2021estimating, grange2021covid}. In contrast, such reduction was not observable in Ukraine after accounting for weather effects, likely due to limited reductions in the operation of CPPs, as detailed in Chapter \ref{chap3}, section \ref{chap3_s1} of our study. However, during the armed conflict with Russia in 2022, numerous CPPs in Ukraine incurred damage, leading to a noticeable decline in NO\textsubscript{2} emissions in densely populated cities. Our findings and evidence suggest that future policies targeting NO\textsubscript{2} reduction from road transportation may not yield comparable effectiveness in Ukraine's populous cities. Comprehensive results for this analysis are presented in Chapter \ref{chap3}, section \ref{chap3_s1}. \par

\textit{2. In what ways did the COVID-19 lockdown influence air quality in Japan, and how can this information serve as evidence and provide suggestions for future policies?} \par
I investigated the influence of NO\textsubscript{2} reduction on O\textsubscript{3} and CH\textsubscript{4} variations in 14 metropolitan areas (MAs) of Japan in 2020, utilizing air quality time series generated by machine learning models under business-as-usual conditions. Additionally, I employed satellite observations and biogeochemical model simulations to analyze changes in air quality. During the lockdown period from April 7 to May 25, 2020, I observed a NO\textsubscript{2} reduction equivalent to a decrease representing 3.4 years and 5 years of the corresponding trends in roadside and ambient air quality recorded from 2010 to 2019. After meteorological normalization, NO\textsubscript{2} decreased by 14.5\% at ambient air stations and 19.1\% at roadside stations. Surprisingly, the NO\textsubscript{2} reduction did not immediately lead to increased O\textsubscript{3}, contrary to what has been widely observed in European cities during the lockdown in 2020 \citep{grange2021covid,shi2021abrupt}. Instead, O\textsubscript{3} levels rose after the lockdown, specifically in August due to favourable sunny conditions. This finding is important for Japan and has not been reported in previous studies. We also found that changes in NO\textsubscript{2} and CO marginally contributed to variations in CH\textsubscript{4} levels across the study areas. \par

We present evidence indicating an increase in O\textsubscript{3} after the lockdown in most of the MAs from Okayama northwards. This occurrence could potentially be attributed to these MAs being VOC-limited areas, implying that future reductions in NO\textsubscript{2} could pose a risk of of increased O\textsubscript{3} levels under favorable sunny conditions. However, in MAs from Hiroshima southwards, instances of NO\textsubscript{x} limitation were observed, implying that future reductions in anthropogenic non-methane volatile organic compounds (NMVOCs) may have minimal effectiveness in lowering O\textsubscript{3} levels \citep{akimoto2022rethinking}. Therefore, based on our findings and evidence, to effectively mitigate the adverse effects on O\textsubscript{3} as well as CH\textsubscript{4}, it is recommended to simultaneously reduce air pollutants, as well as anthropogenic and biogenic volatile organic compounds, in future policies. Detailed results for this analysis are presented in Chapter \ref{chap3}, section \ref{chap3_s2}. \par

\textit{3. What methodologies can be employed to improve PFTs mapping performance in data-sparse regions?} \par
We proposed a combined machine learning approach with a deep convolutional neural network (CNN) which improves the accuracy of PFTs mapping and tree age estimation in Ena city, Japan. First, we employed the Random Forest (RF) classifier using Google Earth Engine (GEE) for forest mapping. Then, we designed a deep CNN architecture that works for PFTs and forest age mapping from coarse and polygonal ground-truth data. The proposed network has U-shape and comprises 3D Atrous Convolutions. The model was optimized by a weighted cross-entropy loss function. We trained the model with times-series Sentinel 1, 2, and Digital Elevation Model data with sparse annotations. Our proposed models achieved 94.5\% overall accuracy (OA) for forest mapping, 77.80\% (OA) for PFTs, and 81.74\% (OA) for forest age classification, respectively which outperformed the 2D and 3D UNET performance. Detailed results for this analysis are presented in Chapter \ref{chap4}, section \ref{chap4_s1}. \par
\textit{4. Can the utilization of updated PFT maps and Transformer-based models enhance the accuracy of global terrestrial carbon flux estimates?} \par
Yes, by utilizing the new PFTs dataset \citep{harper202229} in combination with multivariate timeseries Transformer-based model \citep{zerveas2021transformer} we provided a monthly global gross primary production (GPP) and ecosystem respiration (RECO) dataset from 1990 to 2019 at 0.25° × 0.25° spatial resolution named FluxFormer which outperforms FLUXCOM \citep{jung2019fluxcom}, NIES \citep{zeng2020global}, and MetaFlux \citep{nathaniel2023metaflux} datasets when comparing the correlation at site-level and seasonal pattern with FLUXNET 2015, especially in tropical regions. Additionally, our dataset shows the highest positive trend in GPP from 2001 to 2019, aligning with studies like \citep{{piao2020characteristics, guo2023estimating, yang2022divergent}}. Notably, it captures long-term trends that FLUXCOM and MetaFlux fail to replicate, contradicting the observed significant greening reported by \citep{piao2020characteristics}. Finally, we compare the interannual variations in our dataset with those in other datasets, noting reduced variations in deserts and semi-arid regions compared to the NIES data, given the same remote sensing resources. We find this observation more reasonable due to the extremely low GPP in these areas, which should not lead to high interannual variations. The FluxFormer GPP and RECO products are available at \url{https://doi.org/10.5281/zenodo.10258644}. Detailed results for this analysis are presented in Chapter \ref{chap4}, section \ref{chap4_s2}. \par
\textit{5. How can we efficiently monitor fossil fuels GHGs emissions as well as the capacity of terrestrial forest carbon fluxes, enabling the development of efficient carbon neutrality roadmaps as well as tracking progress at the local level?} \par
We have developed a digital earth platform with a specific focus on providing roadmaps for attaining carbon neutrality at the municipal level in Japan. This platform not only monitors greenhouse gas emissions from fossil fuels and assesses the capacity of local terrestrial forest carbon fluxes but also integrates energy-related data. This encompasses information on energy consumption, electricity statistics, and facilities from major domestic power companies, including data on electricity usage, forecasts, supply, and the distribution of electricity power plants across the country. By consolidating this information, the platform offers a comprehensive overview of the current progress toward achieving a zero-carbon status at the municipal level in Japan. You can access the platform at the following URL: \url{http://de14.digitalasia.chubu.ac.jp/}. Detailed results for this analysis are presented in Chapter \ref{chap5}. \par

\section{Future prospects}
CO\textsubscript{2} is a critical greenhouse gas, but real-time monitoring of fossil fuel CO\textsubscript{2} emissions faces challenges, resulting in high uncertainties in estimated results \citep{marland2008uncertainties}. Traditional bottom-up inventories are time-consuming \citep{marland2008uncertainties}. A recent top-down method uses advancements in satellite observations and data assimilation frameworks. However, current satellites like GOSAT and OCO-2 focus on natural carbon fluxes at regional scales, limiting their ability to quantify anthropogenic emissions \citep{nassar2017quantifying, yang2023using}. In contrast, satellite-derived NO\textsubscript{2} observations, such as OMI or TROPOMI, offer advanced capabilities and higher resolutions. They can potentially constrain fossil fuel CO\textsubscript{2} emissions at city levels. An indirect top-down method leverages NO\textsubscript{2} observations as proxies for fossil fuel CO\textsubscript{2} combustion, beneficial for constraining emissions and monitoring fluctuations \citep{ciais2014current, goldberg2019exploiting}. These approaches, applied in national scales in countries such as the US, Europe, China, and India \citep{konovalov2016estimation, zheng2020satellite, miyazaki2023predictability} and in cities like Wuhan \citep{zhang2023quantifying}, Buenos Aires, Melbourne, and Mexico City \citep{yang2023using}, have not been extensively studied in Japan at national or municipal levels. Conducting such studies in Japan could provide supplementary datasets to refine "bottom-up" inventories and evaluate climate change mitigation strategies. These investigations are necessary for refining our understanding of CO\textsubscript{2} emissions and mitigation strategies, offering valuable insights from local to global scales. \par

Additionally, leveraging the high-frequency temporal data from satellite-derived NO\textsubscript{2} observations for predicting fossil fuel CO\textsubscript{2}, along with the global upscaled terrestrial carbon fluxes detailed in Chapter \ref{chap4}, section \ref{chap4_s2}, is anticipated to provide the most current assessment of achieving zero carbon emissions on both regional and global scales when integrated into a Digital Earth platform as described in Chapter \ref{chap5}. Recent publications in this field within the current year, such as \citep{zhang2023quantifying, yang2023using, miyazaki2023predictability}, reflect active research in these areas. This presents a room for my future works.\par