\begin{center}
 \renewcommand{\thefootnote}{\fnsymbol{footnote}}
 \Large\bfseries \etitle\footnote[1]
 {{\edoctitle}, Graduate School of Engineering\\
 Chubu University, \edate.}
 \renewcommand{\thefootnote}{\arabic{footnote}}
\end{center}
\vspace*{1truemm}
\begin{center}
 \large\eauthor
\end{center}
\vspace*{10truemm}
\begin{center}
 {\bfseries Abstract}
\end{center}
\vspace*{2truemm}
\par

Simultaneously reducing air pollution and greenhouse gases (GHGs) is crucial for combating climate change. Due to their intricate characteristics and interrelation, a comprehensive understanding, along with significant efforts, is necessary to address and diminish these pollutants and gases for future sustainable development on a regional to global scale. This thesis concentrates on three main topics to enhance understanding of regional air pollution characteristics and approaches to monitor carbon neutrality progress at various scales. Initially, this PhD thesis furnishes evidence and recommendations for regional air pollution mitigation policies. It accomplishes this by utilizing multisource data to evaluate extreme intervention events on air quality, which are regarded as real-world practices for reducing anthropogenic activities, as demonstrated by two case studies in Japan and Ukraine. Next, the thesis tackles the gap in estimating seasonal patterns and long-term trends in global terrestrial carbon fluxes, the largest carbon sink that requires accurate quantification to achieve carbon neutrality at both regional and global scales. Finally, the thesis introduces the development of a digital earth platform. This platform incorporates a carbon neutrality roadmap and the monitoring of fossil fuel GHG emissions and forest sinks at a local scale, using Japan's municipalities as a case study. This is assumed to be essential for local policymakers to monitor progress toward achieving carbon neutrality and develop appropriate local roadmaps to facilitate this goal. \par

Based on the experiments conducted during this theis, the most important findings are: \par
\begin{itemize}
    \item we analyzed tropospheric NO\textsubscript{2} levels in Ukraine during two significant periods. We provided evidences that the reduction in road transportaion in Ukraine did not significantly lead to a reduction in NO\textsubscript{2} was not observable in Ukraine after accounting for weather effects, likely due to limited reductions in the operation of Coal Power Plants (CPPs) which is in contrast to Europe where road transport is identified as the primary contributor to NO\textsubscript{2} emissions. However, during the armed conflict with Russia in 2022, numerous CPPs in Ukraine incurred damage, leading to a noticeable decline in NO\textsubscript{2} emissions in densely populated cities. Our findings and evidence suggest that future policies targeting NO\textsubscript{2} reduction from road transportation may not yield comparable effectiveness in Ukraine's populous cities.
    \item We investigated the influence of NO\textsubscript{2} reduction on O\textsubscript{3} and CH\textsubscript{4} variations in 14 metropolitan areas (MAs) of Japan in 2020, utilizing air quality time series generated by machine learning models under business-as-usual conditions. we present evidence indicating an increase in O\textsubscript{3} after the lockdown in most of the MAs from Okayama northwards. This occurrence could potentially be attributed to these MAs being VOC-limited areas, implying that future reductions in NO\textsubscript{2} could pose a risk of of increased O\textsubscript{3} levels under favorable sunny conditions. However, in MAs from Hiroshima southwards, instances of NO\textsubscript{x} limitation were observed, implying that future reductions in anthropogenic non-methane volatile organic compounds (NMVOCs) may have minimal effectiveness in lowering O\textsubscript{3} levels. Therefore, based on our findings and evidence, to effectively mitigate the adverse effects on O\textsubscript{3} as well as CH\textsubscript{4}, it is recommended to simultaneously reduce air pollutants, as well as anthropogenic and biogenic volatile organic compounds, in future policies.
    \item By utilizing the new plant PFTs dataset in combination with multivariate timeseries Transformer-based model, we provided a monthly global gross primary production (GPP) and ecosystem respiration (RECO) dataset from 1990 to 2019 at 0.25° × 0.25° spatial resolution which outperforms FLUXCOM, NIES, and MetaFlux datasets when comparing the correlation at site-level and seasonal pattern with FLUXNET 2015, especially in tropical regions. Additionally, our dataset shows the highest positive trend in GPP from 2001 to 2019, aligning with other widely recognized studies. Notably, it captures positive long-term trends that FLUXCOM and MetaFlux fail to replicate. Finally, we compare the interannual variations in our dataset with those in other datasets, noting reduced variations in deserts and semi-arid regions compared to the NIES data, given the same remote sensing resources. We find this observation more reasonable due to the extremely low GPP in these areas, which should not lead to high interannual variations 
    \item We have developed a digital earth platform with a specific focus on providing roadmaps for attaining carbon neutrality at the municipal level in Japan. This platform not only monitors greenhouse gas emissions from fossil fuels and assesses the capacity of local terrestrial forest carbon fluxes but also integrates energy-related data. This encompasses information on energy consumption, electricity statistics, and facilities from major domestic power companies, including data on electricity usage, forecasts, supply, and the distribution of electricity power plants across the country. By consolidating this information, the platform offers a comprehensive overview of the current progress toward achieving a zero-carbon status at the municipal level in Japan. The platform is available at \url{http://de14.digitalasia.chubu.ac.jp/}.
\end{itemize}

\vspace*{5truemm}
%\vspace*{1truemm}
\begin{flushleft}
 {\bfseries Keywords:}
\end{flushleft}\ekeywords
