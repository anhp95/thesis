\begin{center}
 \renewcommand{\thefootnote}{\fnsymbol{footnote}}
 \Large\bfseries \etitle\footnote[1]
 {{\edoctitle}, Graduate School of Engineering\\
 Chubu University, \edate.}
 \renewcommand{\thefootnote}{\arabic{footnote}}
\end{center}
\vspace*{1truemm}
\begin{center}
 \large\eauthor
\end{center}
\vspace*{10truemm}
\begin{center}
 {\bfseries Abstract}
\end{center}
\vspace*{2truemm}
\par

Simultaneously reducing air pollution and greenhouse gases (GHGs) is crucial for combating climate change. Due to their intricate characteristics and interrelation, a comprehensive understanding, along with significant efforts, is necessary to address and diminish these pollutants and gases for future sustainable development on a regional to global scale. This thesis concentrates on three main topics to enhance understanding of regional air pollution characteristics and approaches to monitor carbon neutrality progress at various scales. Initially, this thesis furnishes evidence and recommendations for regional air pollution mitigation policies. It accomplishes this by utilizing multisource data to evaluate extreme intervention events on air quality, which are regarded as real-world practices for reducing anthropogenic activities, as demonstrated by two case studies in Japan and Ukraine. Next, the thesis tackles the gap in estimating seasonal patterns and long-term trends in global terrestrial carbon fluxes, the largest carbon sink that requires accurate quantification to achieve carbon neutrality at both regional and global scales. Finally, the thesis introduces the development of a digital earth platform. This platform incorporates carbon neutrality roadmaps and the monitoring of fossil fuel GHG emissions and forest sinks at a local scale, using Japan's municipalities as a case study. This is assumed to be essential for local policymakers to monitor progress toward achieving carbon neutrality and develop appropriate local roadmaps to facilitate this goal. \par


Based on the experiments conducted in this thesis, the key findings are as follows:\par

\begin{itemize}
    \item I conducted an analysis of tropospheric NO\textsubscript{2} levels in Ukraine during two significant periods. To isolate the meteorological effects, I utilized a business-as-usual NO\textsubscript{2} time series. My findings provide evidence that the reduction in road transportation in Ukraine did not significantly result in a decrease in NO\textsubscript{2} levels during the Covid-19 lockdown in 2020. This is likely attributed to limited reductions in the operation of Coal Power Plants (CPPs), in contrast to Europe, where road transport is identified as the primary contributor to NO\textsubscript{2} emissions, even after accounting for weather effects. However, during the armed conflict with Russia in 2022, numerous CPPs in Ukraine incurred damage, leading to a noticeable decline in NO\textsubscript{2} emissions in densely populated cities. Based on these findings and evidence, it is suggested that future policies aimed at reducing NO\textsubscript{2} from road transportation may not achieve comparable effectiveness in Ukraine's populous cities.
    \item I examined the impact of NO\textsubscript{2} reduction on O\textsubscript{3} and CH\textsubscript{4} variations in 14 metropolitan areas (MAs) of Japan in 2020. This analysis utilized air quality time series generated by machine learning models under business-as-usual conditions. My findings present evidence of an increase in O\textsubscript{3} levels after the Covid-19 lockdown in most MAs in 2020, ranging from Okayama to the northeast. This phenomenon may be attributed to these MAs being VOC-limited areas, suggesting that future reductions in NO\textsubscript{2} could potentially lead to elevated O\textsubscript{3} levels under favorable sunny conditions. However, in MAs from Hiroshima to the southwest, instances of NO\textsubscript{x} limitation were observed during the summer, indicating that future reductions in anthropogenic non-methane volatile organic compounds may have minimal effectiveness in lowering O\textsubscript{3} levels. Therefore, based on these findings and evidence, I recommend simultaneous reduction of air pollutants, as well as anthropogenic and biogenic volatile organic compounds, in future policies to effectively mitigate adverse effects on both O\textsubscript{3} and CH\textsubscript{4}.
    \item By employing the recently updated dataset of plant functional types (PFTs) in conjunction with a multivariate time series Transformer-based model, I have generated a monthly global dataset of gross primary production (GPP) and ecosystem respiration (RECO) spanning from 1990 to 2019 at a spatial resolution of 0.25° × 0.25° named FluxFormer. This dataset demonstrates superior performance compared to FLUXCOM, NIES, and MetaFlux datasets when assessing correlations at the site level and seasonal patterns with FLUXNET 2015, particularly in tropical regions. Moreover, FluxFormer exhibits the highest positive trend in GPP from 2001 to 2019, aligning with trends derived from dynamic global vegetation models that account for the CO\textsubscript{2} fertilization effect. Notably, it captures positive long-term trends that FLUXCOM and MetaFlux fail to replicate. Finally, I compare the interannual variations in FluxFormer with those in other datasets, observing reduced variations in deserts and semi-arid regions compared to the NIES data, despite utilizing the same remote sensing resources. This observation appears more reasonable, considering the extremely low GPP in these areas, which should not result in high interannual variations. The FluxFormer GPP and RECO products are available at \url{https://zenodo.org/records/10258644}.
    \item I have developed a digital earth platform with a specific emphasis on creating roadmaps for achieving carbon neutrality at the municipal level in Japan. The platform also serves the purpose of monitoring local greenhouse gas emissions from fossil fuels and assessing the capacity of local forest sinks. It incorporates energy-related data, including information on energy consumption and electricity statistics sourced from major domestic power companies. This data encompasses details on electricity usage, forecasts, supply, and the distribution of electricity power plants throughout the country. Through the consolidation of this information, my aim is for the platform to provide a comprehensive overview of the current progress toward attaining a zero-carbon status at the municipal level in Japan. The platform is accessible at \url{http://de14.digitalasia.chubu.ac.jp/}.
\end{itemize}

Building upon the findings and experiments conducted in this thesis, my future works involve researching the utilization of high-frequency temporal data from satellite-derived NO\textsubscript{2} observations for predicting fossil fuel CO\textsubscript{2}, given its recent prominence. In conjunction with the globally upscaled terrestrial carbon fluxes (FluxFormer), my objective is to incorporate monthly predictions of fossil fuel CO\textsubscript{2} into the digital earth platform. This integration will enhance the continuous monitoring of progress toward achieving zero carbon at both the local and regional levels, offering a higher frequency of assessment. \par

\vspace*{5truemm}
%\vspace*{1truemm}
\begin{flushleft}
 {\bfseries Keywords:}
\end{flushleft}\ekeywords
